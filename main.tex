% --------------------------
% ---- DECLARE PACKAGES ----
% --------------------------
\documentclass[a4paper, oneside, openright]{book}
\usepackage[T1]{fontenc} % Font encoding, T1 = it
\usepackage{lmodern}
\usepackage{multicol} % Per il frontespizio
\usepackage[utf8]{inputenc} % Input encoding - per caratteri particolari
\usepackage[english,italian]{babel} % Lingua principale italiano, con parti in inglese
\usepackage{blindtext} % Per la generazione di paragrafi lorem ipsum
\usepackage{graphicx} % Per includere immagini esterne
\usepackage[a4paper,top=2.5cm,bottom=2.5cm,left=2cm,right=2cm]{geometry} %impaginazione e margini documento
\usepackage[fontsize=13pt]{scrextend} %dimensione font
\usepackage{graphicx}
\usepackage[parfill]{parskip} % Disabilita l'indentazione dopo essere andati a capo
% \usepackage[hang,flushmargin]{footmisc} % Disabilita l'indentazione nelle footnotes
\usepackage{titlesec}
\usepackage{minted} % Per i blocchi di codice
\usepackage{float}
\usepackage[font=scriptsize, skip=5pt]{caption} % Spazio tra la caption e l'immagine
\usepackage[backend=biber, style=numeric, backref=true,defernumbers=true]{biblatex}
\usepackage[immediate]{silence}
\WarningFilter[temp]{latex}{Command} % silence the warning
\usepackage{sectsty}
\DeactivateWarningFilters[temp] % So nothing unrelated gets silenced
\usepackage{hyperref} % Rende l'indice cliccabile
\usepackage[justification=centering]{caption} % Per centrare le captions
\usepackage{csquotes} % Dipendenza di babel
\usepackage[bottom]{footmisc} % Posiziona le footnotes alla fine della pagina
\usepackage{mdframed}


% \surroundwithmdframed{minted} 

% \usemintedstyle{fruity} 
\setminted{fontsize=\footnotesize,frame=single}

% ------------------------
% ---- DOCUMENT SETUP ----
% ------------------------
\pagestyle{plain}
\raggedbottom % Se la pagina non è completa, lascia lo spazio alla fine

\titleformat{\chapter}[display]
    % {\normalfont\huge\bfseries}{\chaptertitlename\ \thechapter}{10pt}{\LARGE}
    {\normalfont\huge\bfseries}{}{-75pt}{\LARGE}
    
\titlespacing*{\chapter}{0pt}{0pt}{20pt}
\chaptertitlefont{\fontsize{22pt}{30pt}\selectfont}

\hypersetup{ % Setup dell'aspetto dei link
    colorlinks,
    citecolor=black,
    filecolor=black,
    linkcolor=black,
    urlcolor=black
}

% \renewcommand{\footnoterule}{ % Rende la linea delle footnotes larga tutta la pagina
%   \kern -3pt
%   \hrule width \textwidth height 1pt
%   \kern 2pt
% } 
\renewcommand{\footnotesize}{\fontsize{11pt}{13pt}\selectfont} % Imposta la dimensione del testo delle footnotes
\setlength{\footnotesep}{0.5cm} % Imposta lo spazio fra e singole footnotes
\setlength{\skip\footins}{1.5cm} % Imposta lo spazio fra il corpo e le footnotes

\DeclareUnicodeCharacter{02BC}{}

% ------------------------
% ---- DOCUMENT START ----
% ------------------------
\addbibresource{bibliography.bib} % Importiamo la bibliografia
\begin{document} % Inizio documento
\pagenumbering{gobble} % Disabilita numerazione pagine
\begin{titlepage}
\begin{figure}[!htb]
    \centering
    \includegraphics[width=5cm]{Immagini/uninsubria-logo.png}
\end{figure}

\begin{center}
    \Large{\textbf{UNIVERSITÀ DEGLI STUDI DELL'INSUBRIA}}
    \vspace{3mm}
    \\ \normalsize{DIPARTIMENTO DI SCIENZE TEORICHE E APPLICATE}
    \vspace{6mm}
    \\ \normalsize{CORSO  DI STUDIO TRIENNALE  IN}
    \\  \normalsize{\textbf{INFORMATICA}}
    \vspace{13mm}
    % \\ \normalsize{Tesi  di Laurea}
\end{center}

\vspace{10mm}
\begin{center}
    \LARGE{\textbf{Sviluppo di un sistema embedded per il controllo della temperatura in camera di collaudo.}}
\end{center}

\vspace*{\fill}

\begin{minipage}[t]{1\textwidth}
    \begin{multicols}{2}
    	{\normalsize{\textbf{Relatore:}}{\normalsize\vspace{1mm}
        \\ \normalsize{Prof. Carlo Dossi }}} \\ 

        % {\normalsize{\textbf{Co-relatore:}}{\normalsize\vspace{1mm}
        % \\ \normalsize{Prof. Luca Verdi }}} \\
        
         \columnbreak
         \columnbreak

         \begin{flushright}
            {\normalsize{\textbf{Tesi di Laurea di:}}{\normalsize\vspace{1mm}
            \\ \normalsize{Mattia Papaccioli}\\
            \normalsize{Matricola  747053 }}} \\
         \end{flushright}
    \end{multicols}
\end{minipage}

\begin{center}
    {\normalsize{\textbf{Anno accademico:}}{\normalsize\vspace{1mm}
    \\ \normalsize{2025/2026}}}  
\end{center}

\end{titlepage}
 % PAGINA FRONTESPIZIO
% \cleardoublepage
\thispagestyle{empty}
\vspace*{\stretch{7}}
\begin{flushright}
\itshape Lorem ipsum, dolor sit \\
amet, consectetuer adipiscing elit.  \\ \vspace{5mm}
Praesent imperdiet mi nec ante \\
donec ullamcorper, felis non sodales.
\end{flushright}
\vspace*{\stretch{2}}
\cleardoublepage % PAGINA DEDICA
\tableofcontents  % Genera l'indice
\newpage % Nuova pagna
\pagenumbering{arabic} % Riabilita la numerazione in modo che cominci dal primo capitolo
\setcounter{chapter}{0} % Fa risultare l'introduzione come capitolo 0
% ------------------
% ---- CHAPTERS ----
% ------------------
\chapter{Introduzione}
Lorem Ipsum \cite{defusco}
 dolor sit amet \cite{Axa}.
Consectetuer adipiscing.\footnote{Lorem ipsum dolor sit amet, consectetuer adipiscing elit. Etiam lo-
bortis facilisis sem. Nullam nec mi et neque pharetra sollicitudin. Praesent
imperdiet mi nec ante. }

\begin{minted}{java}
// Hello.java
import javax.swing.JApplet;
import java.awt.Graphics;

public class Hello extends JApplet {
    public void paintComponent(Graphics g) {
        g.drawString("Hello, world!", 65, 95);
    }    
}
\end{minted}

\Blindtext

\begin{figure}
    \centering
    \includegraphics[width=15cm]{Immagini/lvgl-gui.png}  
    \caption{Interfaccia grafica per il controllo della temperatura}
\end{figure}

\chapter{Control}

\section{Graphical User Interface}
Per controllare la temperatura della camera di collaudo, l'operatore imposta la temperatura target mediante un display touchscreen "NOME DISPLAY". L'interfaccia grafica è sviluppata utilizzando la libreria LVGL \cite{LVGL} e si è utilizzato un template \cite{LVGL_LINUX} contenente il porting su Linux fornito dagli sviluppatori della libreria. 

\begin{table}[H]
\begin{minted}{C}
// TODO CHANGEME READ THIS FROM A FILE
static float target_temperature = read_temperature_file(); 
lv_obj_t * screen;
lv_obj_t * target_temperature_label;

const int padding_button = 50;
const int height_button  = 50;
const int width_button   = 50;

static void increment_temperature(lv_event_t * e)
{
    if(lv_event_get_code(e) != LV_EVENT_CLICKED) {
        return;
    }
    target_temperature++;
    lv_label_set_text_fmt(
        target_temperature_label, "%.1f°C", target_temperature
    );
}

static void decrement_temperature(lv_event_t * e)
{
    if(lv_event_get_code(e) != LV_EVENT_CLICKED) {
        return;
    }
    target_temperature--;
    lv_label_set_text_fmt(
        target_temperature_label, "%.1f°C", target_temperature
    );
}
\end{minted}
\caption{Funzioni di callback nel ciclo principale della GUI LVGL}
\end{table}


\begin{table}[H]
\begin{minted}{C}
screen = lv_scr_act();
lv_obj_t * increment_temperature_button = lv_btn_create(screen);
lv_obj_align(
    increment_temperature_button, LV_ALIGN_BOTTOM_RIGHT, 
    -padding_button, -padding_button);
lv_obj_set_height(increment_temperature_button, height_button);
lv_obj_set_width(increment_temperature_button, width_button);
lv_obj_add_event_cb(
    increment_temperature_button, increment_temperature, 
    LV_EVENT_ALL, NULL
);

lv_obj_t * increment_temperature_label = lv_label_create(
    increment_temperature_button
);
lv_label_set_text(increment_temperature_label, "+");
lv_obj_set_style_text_font(
    increment_temperature_label, &lv_font_montserrat_48, 0
);
lv_obj_center(increment_temperature_label);

lv_obj_t * decrement_temperature_button = lv_btn_create(screen);
lv_obj_align(
    decrement_temperature_button, LV_ALIGN_BOTTOM_LEFT, 
    padding_button, -padding_button);
lv_obj_set_height(decrement_temperature_button, height_button);
lv_obj_set_width(decrement_temperature_button, width_button);
lv_obj_add_event_cb(
    decrement_temperature_button, decrement_temperature, 
    LV_EVENT_ALL, NULL
);

lv_obj_t * decrement_temperature_label = lv_label_create(
    decrement_temperature_button
    );
lv_label_set_text(decrement_temperature_label, "-");
lv_obj_set_style_text_font(
    decrement_temperature_label, &lv_font_montserrat_48, 0
);
lv_obj_center(decrement_temperature_label);

target_temperature_label = lv_label_create(screen);
lv_label_set_text_fmt(
    target_temperature_label, "%.1f°C", target_temperature
);
lv_obj_set_style_text_font(
    target_temperature_label, &lv_font_montserrat_48, 0
);
lv_obj_align(target_temperature_label, LV_ALIGN_CENTER, 0, 0);

\end{minted}

\caption{Albero dei widget nel ciclo principale}
\end{table}

I backend utilizzati da LVGL per l'I/O sono libevdev e il framebuffer device. Sono stati scelti per la loro semplicità e il ridotto utilizzo di risorse.

Libevdev \cite{libevdev} è una libreria che gestisce gli eventi di input: riceve i tocchi dal touchscreen e li passa all'interfaccia grafica.

Il framebuffer device è semplicemente il file \texttt{/dev/fb0}, letto dal touchscreen, che contiene il colore di ciascun pixel dello schermo.

\subsection{Compilazione della GUI}

Per la compilazione dell'applicazione è necessaria una toolchain adatta all'architettura ARM. Nel nostro caso, ci affidiamo al compilatore e alle librerie fornite da Buildroot \autoref{EmbeddedLinux:Buildroot}. 


\begin{table}[H]
\begin{minted}[fontsize=\small]{C}
set(CMAKE_SYSTEM_NAME Linux)
set(CMAKE_SYSTEM_PROCESSOR arm)

set(tools ~/buildroot/output/host/bin/arm-buildroot-linux-gnueabihf-)
set(CMAKE_C_COMPILER ${tools}gcc)
set(CMAKE_CXX_COMPILER ${tools}g++)

set(EVDEV_INCLUDE_DIRS ~/buildroot/output/staging/usr/include/libevdev/)
set(EVDEV_LIBRARIES ~/buildroot/output/staging/usr/lib/libevdev.so)

set(BUILD_SHARED_LIBS ON)    
\end{minted}
\caption{\texttt{cross\_compile\_setup.cmake}}
\end{table}


Il comando \texttt{cmake -DCMAKE\_TOOLCHAIN\_FILE=./cross\_compile\_setup.cmake -B build -S .} genera i Makefile necessari per la cross-compilazione, che vengono poi eseguiti con \texttt{make -C build -j}.

LVGL viene compilata come libreria condivisa, mentre l'applicazione come eseguibile.

\begin{figure}[H]
    \centering
    \includegraphics[width=15cm]{Immagini/lvgl-gui.png}  
    \caption{Interfaccia grafica per il controllo della temperatura}
\end{figure}




\section{Admin Control}

\section{Logging and Monitoring}
\chapter{Embedded Linux}


% Vada per l'opzione a) Nel frattempo ho realizzato che il supporto per
% Ganador rev.4 era stato completato dal collega Giandomenico, chiedo
% venia è un po' che non tocco questo progetto. Quindi un problema in meno.

% Il sistema embedded che ti forniremo è costituito da una host-board
% chiamata Ganador (rev.4) e un system-on-module (SoM) chiamato
% Vulcano-A5. A completare il tutto un display da 7" connesso tramite
% l'interfaccia TTL e dotato di touchscreen resistivo.

% Allego i diagrammi a blocchi di SoM e host-board.

% Nel file vulcano-sw.pdf allegato trovi della documentazione riguardo
% l'intero sistema che si compone di varie parti, buildroot, AT91boot
% (bootloader di secondo livello), Barebox (bootloader di terzo livello) e
% Linux.

% Quando fu scritta quella documentazione Linux, AT91boot e Barebox
% venivano compilati esternamente da buildroot ma sempre tramite il
% cross-compilatore compilato da buildroot.

% Giandomenico in occasione del porting di Vulcano-A5 su Ganador ha
% integrato tutto quanto in buildroot
% (git.amelchem.com:amel/buildroot.git), il suo branch è vulcano-uni ma
% non l'ho mai revisionato in quanto il progetto Ganador si era
% temporaneamente arenato. Ho appena sistemato i commit e fatto push nel
% branch "Vulcano", dovrebbe essere tutto in ordine ma non ho compilato
% quindi non ci metto la mano sul fuoco.

% Dovresti essere in grado di compilare tutto quanto ti serve con:

% git clone git://git.buildroot.net/buildroot
% cd buildroot
% git remote add amel git@git.amelchem.com:amel/buildroot.git
% git fetch amel
% git checkout Vulcano

% make vulcanoa5_defconfig
% make -j$(nproc)

% Armati di pazienza perché partirà una lunga compilazione, prima degli
% host tool, quindi tutto quanto ti servirà per cross-compilare (gcc e
% libc in primis), poi del kernel, busybox e i pochi altri pacchetti
% abilitati.

% Questo il manuale di buildroot:
% https://buildroot.org/downloads/manual/manual.html

% Bada che nel defconfig di vulcano viene abilitato qt5 come dipendenza di
% una piccola applicazione dimostrativa: amel-qt5app. A te non serve, puoi
% disabilitare l'intera libreria QT andando ad eliminare le varie righe
% QT5BASE_ dal tuo .config locale o tramite make menuconfig, consiglio
% definisciti un tuo defconfig a partire da vulcanoa5_defconfig con tutti
% i pacchetti che ti servono (vedi paragrafo della documentazione 9.3.
% Storing the Buildroot configuration). Recupererai diversi minuti di
% compilazione spegnendo qt5.

% Giunti a questo punto dovrai aggiungere il pacchetto della tua
% applicazione a buildroot, vedi a tal proposito la documentazione di
% buildroot al paragrafo:
% https://buildroot.org/downloads/manual/manual.html#adding-packages. Vedi
% anche esempio amel-qt5app con git show 01e31e1fc4

% Nel frattempo ti faccio preparare un kit con Ganador, Vulcano-A5,
% Display e sonda termica che basata su DS18B20 il quale parla 1-wire ma
% lo collegheremo tramite un adattatore USB a Ganador perché purtroppo non
% ci sono GPIO bidirezionali del SoC esposti direttamente altrimenti si
% sarebbe potuto usare uno di quelli.

% Tutto chiaro?

% Non spaventarti, per dubbi o perplessità scrivimi.


\section{Hardware}
La scheda utilizzata per il sistema embedded è sviluppata da AMEL e comprende:
\begin{itemize}
    \item una host-board Ganador (rev. 4);
    \item un system-on-module Vulcano-A5;
    \item una CPU ARM;
    \item una memoria SD utilizzata come disco fisso;
    \item un'interfaccia Ethernet;
    \item un'interfaccia seriale;
    \item un display touchscreen.
\end{itemize}

\section{Costruzione del sistema}
Per orchestrare il sistema è stato utilizzato Linux. È stato creato un kernel personalizzato con soli i moduli essenziali per il funzionamento, data la limitatezza delle risorse hardware. Lo strumento utilizzato per completare l'opera è Buildroot \cite{buildroot}, che consiste essenzialmente in una serie di Makefile per installare e cross-compilare tutte le librerie e i pacchetti necessari alla costruzione e all'esecuzione del sistema. Inoltre, si occupa di creare il filesystem e di prepararlo in un'immagine pronta per essere scritta sulla scheda SD del sistema embedded.

Per aggiungere l'interfaccia grafica sviluppata con LVGL è stato necessario creare un nuovo pacchetto in Buildroot.

All'avvio del sistema, il bootloader del chip attiva AT91bootstrap, che a sua volta avvia Barebox, il quale carica il kernel in memoria.

\subsection{Installazione di un pacchetto in Buildroot}
\label{EmbeddedLinux:Buildroot}
Per aggiungere un pacchetto a Buildroot è necessario inserire una nuova voce nella cartella \texttt{package}, comprendente un file \texttt{Config.in} e un file \texttt{.mk}.

Questi due file contengono le istruzioni che consentono a Buildroot di risolvere le dipendenze, scaricare e installare il pacchetto nel filesystem del dispositivo target.


\newpage

\begin{table}
\begin{minted}{bash}
AMEL_TEMP_CONTROL_VERSION = 44c17c6f2c492f1f3c7d8a6767df390c8d13eb9c
AMEL_TEMP_CONTROL_SITE = git@git.amelchem.com:mpapaccioli/temp-control.git
AMEL_TEMP_CONTROL_SITE_METHOD = git

AMEL_TEMP_CONTROL_DEPENDENCIES = libevdev
AMEL_TEMP_CONTROL_GIT_SUBMODULES = YES

define AMEL_TEMP_CONTROL_BUILD_CMDS
	cmake -DCMAKE_TOOLCHAIN_FILE=$(@D)/user_cross_compile_setup.cmake \
        -B $(@D)/build -S $(@D)
	make -C $(@D)/build -j

endef

define AMEL_TEMP_CONTROL_INSTALL_TARGET_CMDS
 	$(INSTALL) -d $(TARGET_DIR)/opt/amel-temp-control/
	cp $(@D)/build/bin/lvglsim $(TARGET_DIR)/opt/amel-temp-control/main
	cp -r $(@D)/build/lvgl/lib/* $(TARGET_DIR)/usr/lib
	
endef

$(eval $(generic-package))
\end{minted}
\caption{\texttt{amel-temp-control.mk}}
\end{table}

Inizialmente, viene clonata la repository temp-control, contenente la GUI, al commit specificato, inizializzando i sottomoduli e verificando la presenza della dipendenza \texttt{libevdev}.

Successivamente, vengono cross-compilate la libreria LVGL e l'applicazione con interfaccia grafica utilizzando il compilatore ARM fornito da Buildroot.

Infine, i binari della libreria e l'eseguibile dell'applicazione vengono installati sulla macchina target.

Dopo aver ripetuto questo processo per tutti i pacchetti desiderati, il filesystem e l'immagine del kernel vengono assemblati in un file \texttt{sdcard.img} pronto per essere scritto su una scheda SD e avviato sul dispositivo embedded.


\section{File I/O}

\section{PID Control}


\section{MODBUS RTU}
% \blinddocument % Genera paragrafi e contenuti placeholder
\input{Capitoli/Conclusione}

% ----------------------
% ---- BIBLIOGRAPHY ----
% ----------------------
\backmatter
\sloppy % serve a non far andare i link oltre ai margini
\printbibliography[heading=bibintoc, title=Bibliografia]
% \printbibliography[type=online, heading=bibintoc, title=Sitografia]

% ----------------------
% ---- DOCUMENT END ----
% ----------------------
\end{document} % Fine documento
