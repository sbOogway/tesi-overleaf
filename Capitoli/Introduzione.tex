\chapter{Introduzione}
\section{Abstract}
La camera di collaudo, situata presso AMEL s.r.l., è utilizzata per testare carichi resistivi. Questi producono calore, rendendo necessario il raffreddamento dell'ambiente, soprattutto nei mesi più caldi.

Il sistema embedded controlla la frequenza di una ventola di raffreddamento mediante un controllo PID:
\begin{enumerate}
    \item La temperatura ambiente viene letta tramite un sensore.
    \item Questa viene utilizzata come ingresso del sistema PID.
    \item L'algoritmo PID calcola la tensione da inviare a un inverter, il quale determina la frequenza della ventola in base all'errore attuale (azione proporzionale) e a quello accumulato (azione integrativa).
    \item Questo sistema di retroazione negativa viene applicato continuamente per mantenere costante la temperatura ambiente.
\end{enumerate}

La comunicazione tra l'inverter e il sistema embedded avviene tramite il protocollo MODBUS RTU.

Il sistema embedded fornisce inoltre un'interfaccia per regolare la temperatura target dell'ambiente mediante un display touchscreen. La GUI è sviluppata utilizzando la libreria LVGL.

Il sistema embedded è collegato alla rete aziendale tramite Ethernet; attraverso un web server sarà possibile regolare la temperatura target anche da remoto.

La comunicazione tra il sistema embedded e il web server avviene mediante scrittura su file o IPC (Inter-Process Communication).

È ancora da valutare l'impiego di un database per registrare la temperatura nel tempo. In tal caso, il web server gestirà l'interazione con esso.


\newpage

\begin{figure}[!h]
    \centering
    \includegraphics[width=21cm, angle=90]{Immagini/system-uml.drawio.png}  
    \caption{Diagramma UML che illustra il sistema}
\end{figure}

\newpage 

\section{Struttura del codice sorgente}
Il codice sorgente del progetto è pubblicato su GitHub \cite{root}. Questa repository è composta da più sottomoduli:

\begin{itemize}
    \item tesi-gui \cite{GUI}
    \item tesi-overleaf \cite{LaTeX_Source}
    \item tesi-modbus \cite{}
\end{itemize}

Questo approccio è stato adottato per garantire modularità e organizzazione del software \cite{git_submodules}.