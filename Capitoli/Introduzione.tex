\chapter{Introduzione}
\section{Abstract}
La camera di collaudo si trova presso AMEL s.r.l. ed è utilizzata per testare dei carichi resistivi. Essi producono calore ed è necessario raffreddare l'ambiente, sopratutto nei mesi più caldi.

Il sistema embedded si occupa di controllare la frequenza di una ventola di raffreddamento tramite un controllo PID:
\begin{enumerate}
    \item Viene letta la temperatura nella stanza tramite un sensore
    \item Essa viene utilizzata come input del sistema PID 
    \item L'equazione PID restituisce in output la tensione da mandare ad un inverter che a sua volta restituisce in output la frequenza della ventola considerando l'errore attuale (azione proporzionale) e quello passato (azione integrativa).
    \item Questo sistema di retroazione negativa viene continuamente attuato per mantenere costante la temperatura nella stanza.
\end{enumerate}

La comunicazione tra l'inverter e il sistema embedded è effettuata tramite il protocollo MODBUS RTU.

Il sistema embedded si occupa inoltre di fornire un interfaccia per regolare la temperatura target della stanza tramite display touchscreen. La GUI è programmata con la libreria LVGL.

Il sistema embedded è collegato alla rete aziendale tramite ethernet e tramite un web server sara possibile regolare la temperatura target anche da remoto.

La comunicazione tra sistema embedded e web server avviene attraverso scrittura su file/IPC.

E ancora da valutare se verra utilizzato un database per loggare la temperatura nel tempo. Nel caso il web server si occupera di interagire con esso.


\newpage

\begin{figure}[!h]
    \centering
    \includegraphics[width=21cm, angle=90]{Immagini/system-uml.drawio.png}  
    \caption{Diagramma UML che illustra il sistema}
\end{figure}

\newpage 

\section{Struttura del codice sorgente}
Il codice sorgente del progetto è pubblicato su Github\cite{root}. Questa repository è composta da più submodules:

\begin{itemize}
    \item tesi-gui \cite{GUI}
    \item tesi-overleaf \cite{LaTeX_Source}
    \item tesi-modbus \cite{}
\end{itemize}

È stato utilizzato questo pattern per garantire modularità ed organizzazione del software\cite{git_submodules}.