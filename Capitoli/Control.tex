\chapter{Control}

\section{Graphical User Interface}
Per controllare la temperatura della camera di collaudo, l'operatore imposta la temperatura target mediante un display touchscreen "NOME DISPLAY". L'interfaccia grafica è sviluppata utilizzando la libreria LVGL \cite{LVGL} e si è utilizzato un template \cite{LVGL_LINUX} contenente il porting su Linux fornito dagli sviluppatori della libreria. 

\begin{table}[H]
\begin{minted}{C}
// TODO CHANGEME READ THIS FROM A FILE
static float target_temperature = read_temperature_file(); 
lv_obj_t * screen;
lv_obj_t * target_temperature_label;

const int padding_button = 50;
const int height_button  = 50;
const int width_button   = 50;

static void increment_temperature(lv_event_t * e)
{
    if(lv_event_get_code(e) != LV_EVENT_CLICKED) {
        return;
    }
    target_temperature++;
    lv_label_set_text_fmt(
        target_temperature_label, "%.1f°C", target_temperature
    );
}

static void decrement_temperature(lv_event_t * e)
{
    if(lv_event_get_code(e) != LV_EVENT_CLICKED) {
        return;
    }
    target_temperature--;
    lv_label_set_text_fmt(
        target_temperature_label, "%.1f°C", target_temperature
    );
}
\end{minted}
\caption{Funzioni di callback nel ciclo principale della GUI LVGL}
\end{table}


\begin{table}[H]
\begin{minted}{C}
screen = lv_scr_act();
lv_obj_t * increment_temperature_button = lv_btn_create(screen);
lv_obj_align(
    increment_temperature_button, LV_ALIGN_BOTTOM_RIGHT, 
    -padding_button, -padding_button);
lv_obj_set_height(increment_temperature_button, height_button);
lv_obj_set_width(increment_temperature_button, width_button);
lv_obj_add_event_cb(
    increment_temperature_button, increment_temperature, 
    LV_EVENT_ALL, NULL
);

lv_obj_t * increment_temperature_label = lv_label_create(
    increment_temperature_button
);
lv_label_set_text(increment_temperature_label, "+");
lv_obj_set_style_text_font(
    increment_temperature_label, &lv_font_montserrat_48, 0
);
lv_obj_center(increment_temperature_label);

lv_obj_t * decrement_temperature_button = lv_btn_create(screen);
lv_obj_align(
    decrement_temperature_button, LV_ALIGN_BOTTOM_LEFT, 
    padding_button, -padding_button);
lv_obj_set_height(decrement_temperature_button, height_button);
lv_obj_set_width(decrement_temperature_button, width_button);
lv_obj_add_event_cb(
    decrement_temperature_button, decrement_temperature, 
    LV_EVENT_ALL, NULL
);

lv_obj_t * decrement_temperature_label = lv_label_create(
    decrement_temperature_button
    );
lv_label_set_text(decrement_temperature_label, "-");
lv_obj_set_style_text_font(
    decrement_temperature_label, &lv_font_montserrat_48, 0
);
lv_obj_center(decrement_temperature_label);

target_temperature_label = lv_label_create(screen);
lv_label_set_text_fmt(
    target_temperature_label, "%.1f°C", target_temperature
);
lv_obj_set_style_text_font(
    target_temperature_label, &lv_font_montserrat_48, 0
);
lv_obj_align(target_temperature_label, LV_ALIGN_CENTER, 0, 0);

\end{minted}

\caption{Albero dei widget nel ciclo principale}
\end{table}

I backend utilizzati da LVGL per l'I/O sono libevdev e il framebuffer device. Sono stati scelti per la loro semplicità e il ridotto utilizzo di risorse.

Libevdev \cite{libevdev} è una libreria che gestisce gli eventi di input: riceve i tocchi dal touchscreen e li passa all'interfaccia grafica.

Il framebuffer device è semplicemente il file \texttt{/dev/fb0}, letto dal touchscreen, che contiene il colore di ciascun pixel dello schermo.

\subsection{Compilazione della GUI}

Per la compilazione dell'applicazione è necessaria una toolchain adatta all'architettura ARM. Nel nostro caso, ci affidiamo al compilatore e alle librerie fornite da Buildroot \autoref{EmbeddedLinux:Buildroot}. 


\begin{table}[H]
\begin{minted}[fontsize=\small]{C}
set(CMAKE_SYSTEM_NAME Linux)
set(CMAKE_SYSTEM_PROCESSOR arm)

set(tools ~/buildroot/output/host/bin/arm-buildroot-linux-gnueabihf-)
set(CMAKE_C_COMPILER ${tools}gcc)
set(CMAKE_CXX_COMPILER ${tools}g++)

set(EVDEV_INCLUDE_DIRS ~/buildroot/output/staging/usr/include/libevdev/)
set(EVDEV_LIBRARIES ~/buildroot/output/staging/usr/lib/libevdev.so)

set(BUILD_SHARED_LIBS ON)    
\end{minted}
\caption{\texttt{cross\_compile\_setup.cmake}}
\end{table}


Il comando \texttt{cmake -DCMAKE\_TOOLCHAIN\_FILE=./cross\_compile\_setup.cmake -B build -S .} genera i Makefile necessari per la cross-compilazione, che vengono poi eseguiti con \texttt{make -C build -j}.

LVGL viene compilata come libreria condivisa, mentre l'applicazione come eseguibile.

\begin{figure}[H]
    \centering
    \includegraphics[width=15cm]{Immagini/lvgl-gui.png}  
    \caption{Interfaccia grafica per il controllo della temperatura}
\end{figure}




\section{Admin Control}

\section{Logging and Monitoring}